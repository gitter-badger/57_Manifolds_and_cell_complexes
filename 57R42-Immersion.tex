\documentclass[12pt]{article}
\usepackage{pmmeta}
\pmcanonicalname{Immersion}
\pmcreated{2013-03-22 12:35:04}
\pmmodified{2013-03-22 12:35:04}
\pmowner{bshanks}{153}
\pmmodifier{bshanks}{153}
\pmtitle{immersion}
\pmrecord{8}{32834}
\pmprivacy{1}
\pmauthor{bshanks}{153}
\pmtype{Definition}
\pmcomment{trigger rebuild}
\pmclassification{msc}{57R42}
\pmrelated{Submersion}
\pmdefines{closed immersion}

% this is the default PlanetMath preamble.  as your knowledge
% of TeX increases, you will probably want to edit this, but
% it should be fine as is for beginners.

% almost certainly you want these
\usepackage{amssymb}
\usepackage{amsmath}
\usepackage{amsfonts}

% used for TeXing text within eps files
%\usepackage{psfrag}
% need this for including graphics (\includegraphics)
%\usepackage{graphicx}
% for neatly defining theorems and propositions
%\usepackage{amsthm}
% making logically defined graphics
%%%\usepackage{xypic}

% there are many more packages, add them here as you need them

% define commands here
\begin{document}
Let $X$ and $Y$ be manifolds, and let $f$ be a mapping $f: X \to Y$. Choose $x \in X$, and let $y=f(x)$.  Recall that $df_x: T_x(X) \to T_y(Y)$ is the derivative of $f$ at $x$, and $T_z(Z)$ is the tangent space of manifold $Z$ at point $z$.

If $df_x$ is injective, then $f$ is said to be an \emph{immersion at x}. If $f$ is an immersion at every point, it is called an \emph{immersion}.

If the image of $f$ is also closed, then $f$ is called a \emph{closed immersion}. 

The notion of \PMlinkname{closed immersion}{ClosedImmersion} for schemes is the analog of this notion in algebraic geometry.
%%%%%
%%%%%
\end{document}
