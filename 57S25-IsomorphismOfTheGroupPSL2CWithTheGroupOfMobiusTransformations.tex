\documentclass[12pt]{article}
\usepackage{pmmeta}
\pmcanonicalname{IsomorphismOfTheGroupPSL2CWithTheGroupOfMobiusTransformations}
\pmcreated{2013-03-22 12:43:30}
\pmmodified{2013-03-22 12:43:30}
\pmowner{rspuzio}{6075}
\pmmodifier{rspuzio}{6075}
\pmtitle{isomorphism of the group PSL_2(C) with the group of M\"obius transformations}
\pmrecord{9}{33023}
\pmprivacy{1}
\pmauthor{rspuzio}{6075}
\pmtype{Result}
\pmcomment{trigger rebuild}
\pmclassification{msc}{57S25}

\endmetadata

% this is the default PlanetMath preamble.  as your knowledge
% of TeX increases, you will probably want to edit this, but
% it should be fine as is for beginners.

% almost certainly you want these
\usepackage{amssymb}
\usepackage{amsmath}
\usepackage{amsfonts}

% used for TeXing text within eps files
%\usepackage{psfrag}
% need this for including graphics (\includegraphics)
%\usepackage{graphicx}
% for neatly defining theorems and propositions
%\usepackage{amsthm}
% making logically defined graphics
%%%\usepackage{xypic}

% there are many more packages, add them here as you need them

% define commands here

\newcommand{\Prob}[2]{\mathbb{P}_{#1}\left\{#2\right\}}

\newcommand{\smfour}[4]
{
  {
    \left(
      \begin{smallmatrix}
      {#1}&{#2}\\
      {#3}&{#4}
      \end{smallmatrix}
    \right)
  }
}

\newcommand{\mobius}[5]{{\frac{#2#1+#3}{#4#1+#5}}}
\begin{document}
%\newcommand{smfour}[4]{{\left(\begin{smallmatrix}{#1}&{#2}\\{#3}&{#4}\end{smallmatrix}\right)}}
%\newcommand{mobius}[5]{{\frac{#2#1+#3}{#4#1+#5}}}

We identify the group $G$ of M\"obius transformations with the projective special linear group $PSL_2(\mathbb{C})$.  The isomorphism~$\Psi$ (of topological groups) is given by $\Psi: \left[\smfour{a}{b}{c}{d}\right] \mapsto \mobius{z}{a}{b}{c}{d}$.  (Here, the notation $[M]$ means the equivalence class $[M] = \{ Mt \mid t \in \mathbb{C} \}$)

This mapping is:
\begin{description}

\item[Well-defined:]
If $\left[\smfour{a}{b}{c}{d}\right]=\left[\smfour{a'}{b'}{c'}{d'}\right]$
then $(a',b',c',d')=t(a,b,c,d)$ for some $t$, so $z\mapsto\mobius{z}{a}{b}{c}{d}$ is the same transformation as $z\mapsto\mobius{z}{a'}{b'}{c'}{d'}$.

\item[A homomorphism:]
Calculating the composition
\[
\left.\mobius{z}{a}{b}{c}{d}\right|_{z=\mobius{w}{e}{f}{g}{h}} =
\frac{a\mobius{w}{e}{f}{g}{h}+b}{c\mobius{w}{e}{f}{g}{h}+d} =
\frac{(ae+bg)w+(af+bh)}{(ce+dg)w+(cf+dh)}
\]
we see that $\Psi\left(\left[\smfour{a}{b}{c}{d}\right]\right)\cdot \Psi\left(\left[\smfour{e}{f}{g}{h}\right]\right) =
\Psi\left(\left[\smfour{a}{b}{c}{d}\right]\cdot
\left[\smfour{e}{f}{g}{h}\right]\right)$.

\item[A monomorphism:]
If $\Psi\left(\left[\smfour{a}{b}{c}{d}\right]\right)=
\Psi\left(\left[\smfour{a'}{b'}{c'}{d'}\right]\right)$, then it follows that $(a',b',c',d')=t(a,b,c,d)$,
so that
$\left[\smfour{a}{b}{c}{d}\right]=
\left[\smfour{a'}{b'}{c'}{d'}\right]$.

\item[An epimorphism:]
Any M\"obius transformation $z\mapsto\mobius{z}{a}{b}{c}{d}$ is the image $\Psi\left(\left[\smfour{a}{b}{c}{d}\right]\right)$.

\end{description}
%%%%%
%%%%%
\end{document}
