\documentclass[12pt]{article}
\usepackage{pmmeta}
\pmcanonicalname{ThomSpace}
\pmcreated{2013-03-22 15:40:46}
\pmmodified{2013-03-22 15:40:46}
\pmowner{antonio}{1116}
\pmmodifier{antonio}{1116}
\pmtitle{Thom space}
\pmrecord{5}{37620}
\pmprivacy{1}
\pmauthor{antonio}{1116}
\pmtype{Definition}
\pmcomment{trigger rebuild}
\pmclassification{msc}{57R22}
\pmclassification{msc}{55R25}
\pmdefines{Thom space}
\pmdefines{disk bundle}
\pmdefines{sphere bundle}

% used for TeXing text within eps files
%\usepackage{psfrag}
% need this for including graphics (\includegraphics)
%\usepackage{graphicx}
% for neatly defining theorems and propositions
%\usepackage{amsthm}
% making logically defined graphics
%%%\usepackage{xypic}

\usepackage{theorem}
\usepackage{amsmath}
\usepackage{amsfonts}
\usepackage{amssymb}
\newcommand{\limv}[2]{\lim\limits_{#1\rightarrow #2}}
\newcommand{\eb}{\mathbf{e}} % Standard basis
\newcommand{\comp}{\circ} % Function composition
\newcommand{\R}{{\mathbb R}} % The reals
\newcommand{\reals}{{\mathbb R}} % The reals
\newcommand{\integs}{{\mathbb Z}} % The integers
\newcommand{\cpxs}{{\mathbb C}} % The "complexes" :)
\newcommand{\setc}[2]{\left\{#1:\: #2\right\}}
\newcommand{\set}[1]{{\left\{#1\right\}}}
\newcommand{\cycle}[1]{\left(#1\right)}
\newcommand{\tuple}[1]{\left(#1\right)}
\newcommand{\Partial}[2]{\frac{\partial #1}{\partial #2}}
\newcommand{\PartialSl}[2]{\partial #1/\partial #2}
\newcommand{\funcsig}[2]{#1\rightarrow #2}
\newcommand{\funcdef}[3]{#1:\funcsig{#2}{#3}}
\newcommand{\supp}{\mathop{\mathrm{Supp}}} % Support of a function
\newcommand{\sgn}{\mathop{\mathrm{sgn}}} % Sign function
\newcommand{\tr}[1]{#1^\mathrm{tr}} % Transpose of a matrix
\newcommand{\inprod}[2]{\left<#1,#2\right>} % Inner product
\newenvironment{smallbmatrix}{\left[\begin{smallmatrix}}{\end{smallmatrix}\right]}
\newcommand{\maps}[2]{\mathop{\mathrm{Maps}}\left(#1,#2\right)}
\newcommand{\intoc}[2]{\left(#1,#2\right]}
\newcommand{\intco}[2]{\left[#1,#2\right)}
\newcommand{\intoo}[2]{\left(#1,#2\right)}
\newcommand{\intcc}[2]{\left[#1,#2\right]}
\newcommand{\transv}{\pitchfork}
\newcommand{\pair}[2]{\left\langle#1,#2\right\rangle}
\newcommand{\norm}[1]{\left\|#1\right\|}
\newcommand{\sqnorm}[1]{\left\|#1\right\|^2}
\newcommand{\bdry}{\partial}
\newcommand{\inv}[1]{#1^{-1}}
\newcommand{\tensor}{\otimes}
\newcommand{\bigtensor}{\bigotimes}
\newcommand{\im}{\operatorname{im}}
\newcommand{\coker}{\operatorname{im}}
\newcommand{\map}{\operatorname{Map}}
\newcommand{\crit}{\operatorname{Crit}}
\newcommand{\Th}{\operatorname{Th}}
\theorembodyfont{\upshape}
\newtheorem{thm}{Theorem}
\newtheorem{dthm}[thm]{Desired Theorem}
\newtheorem{cor}[thm]{Corollary}
\newtheorem{dcor}[thm]{Desired Corollary}
\newtheorem{lem}[thm]{Lemma}
\newtheorem{prop}[thm]{Proposition}
\newtheorem{defn}{Definition}
\newtheorem{rmk}{Remark}
\newtheorem{exm}{Example}
\newcommand{\cross}{\times}
\newcommand{\del}{\nabla}
\newcommand{\homeo}{\cong}
\newcommand{\isom}{\cong}
\newcommand{\htpyeq}{\backsimeq}
\newcommand{\codim}{\operatorname{codim}}
\newcommand{\projp}{{\mathbb R}P}

% open cells (not very nice...)
\newcommand{\oce}{\smash{\overset{\circ}e}} 
\newcommand{\ocD}{\smash{\overset{\circ}D}} 

\newcommand{\susp}{\Sigma}
\newcommand{\restr}[2]{{#1}|_{#2}}

\renewcommand{\hom}{\mathop{\mathrm{Hom}}} % Homomorphisms functor
\newcommand{\rp}{\reals P} % real projective space
\newcommand{\cp}{\cpxs P} % complex projective space
\newcommand{\zmod}[1]{\integs / #1\integs} % Z/nZ
\begin{document}
Let $\xi\to X$ be a vector bundle over a topological space $X$. Assume that $\xi$ has a Riemannian metric. We can form its associated disk bundle $D(\xi)$ and its associated sphere bundle $S(\xi)$, by letting 
$$ D(\xi) = \{v\in\xi: \|v\|\le 1\}, \quad S(\xi) = \{v\in\xi: \|v\|= 1\}. $$

The \emph{Thom space} of $\xi$ is defined to be the quotient space $D(\xi)/S(\xi)$, obtained by taking the disk bundle and collapsing the sphere bundle to a point. Notice that this makes the Thom space naturally into a based topological space.

Two common forms of notation for the Thom space are $\Th(\xi)$ and $X^\xi$.

\begin{rmk}
If $\xi= X\cross \R^d$ is a trivial vector bundle, then its Thom space is homeomorphic to $\Sigma^d X_+$, where $X_+$ stands for $X$ with an added disjoint basepoint, and $\Sigma^d$ stands for the based suspension iterated $d$ times. Thus, we may think of $X^\xi$ as a ``twisted suspension'' of $X_+$.
\end{rmk}

\begin{rmk}
If $X$ is compact, then $X^\xi$ is homeomorphic as a based space to the one-point compactification of $\xi$. Even if $X$ is not compact, $X^\xi$ can be obtained by doing a one-point compactification on each fiber and then collapsing the resulting section of points at infinity to a point.
\end{rmk}

\begin{rmk}
The choice of Riemannian metric on $\xi$ does not change the homeomorphism type of $X^\xi$, and, by the previous remark, the Thom space can be described without reference to associated disk and sphere bundles.
\end{rmk}
%%%%%
%%%%%
\end{document}
