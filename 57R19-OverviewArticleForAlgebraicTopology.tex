\documentclass[12pt]{article}
\usepackage{pmmeta}
\pmcanonicalname{OverviewArticleForAlgebraicTopology}
\pmcreated{2013-03-22 19:15:48}
\pmmodified{2013-03-22 19:15:48}
\pmowner{bci1}{20947}
\pmmodifier{bci1}{20947}
\pmtitle{Overview article for algebraic topology}
\pmrecord{17}{42193}
\pmprivacy{1}
\pmauthor{bci1}{20947}
\pmtype{Topic}
\pmcomment{trigger rebuild}
\pmclassification{msc}{57R19}
\pmclassification{msc}{57N65}
\pmclassification{msc}{11F23}
\pmclassification{msc}{11E72}
\pmclassification{msc}{18-00}
\pmclassification{msc}{55N30}
\pmclassification{msc}{55N15}
\pmclassification{msc}{55N99}
\pmclassification{msc}{55N40}
\pmclassification{msc}{55N20}
\pmclassification{msc}{55-01}
%\pmkeywords{algebraic topology}
%\pmkeywords{nonabelian}
%\pmkeywords{non-Abelian}
%\pmkeywords{nonabelian algebraic topology}
%\pmkeywords{homotopy}
%\pmkeywords{homology}
%\pmkeywords{groupoid}
%\pmkeywords{double groupoid}
%\pmkeywords{algebroid}
%\pmkeywords{algebraic topology invariants}
\pmrelated{groupoid}
\pmrelated{category}
\pmrelated{GroupoidCategory}
\pmrelated{topology}
\pmrelated{HomotopyDoubleGroupoidOfAHausdorffSpace}
\pmrelated{QuantumGeometry}
\pmrelated{TopologicalSpace}
\pmrelated{HigherDimensionalAlgebra}

\endmetadata

% this is the default PlanetMath preamble. as your 
\usepackage{amsmath, amssymb, amsfonts, amsthm, amscd, latexsym}
%%\usepackage{xypic}
\usepackage[mathscr]{eucal}
% define commands here
\theoremstyle{plain}
\newtheorem{lemma}{Lemma}[section]
\newtheorem{proposition}{Proposition}[section]
\newtheorem{theorem}{Theorem}[section]
\newtheorem{corollary}{Corollary}[section]
\theoremstyle{definition}
\newtheorem{definition}{Definition}[section]
\newtheorem{example}{Example}[section]
%\theoremstyle{remark}
\newtheorem{remark}{Remark}[section]
\newtheorem*{notation}{Notation}
\newtheorem*{claim}{Claim}
\renewcommand{\thefootnote}{\ensuremath{\fnsymbol{footnote%%@
}}}
\numberwithin{equation}{section}
\newcommand{\Ad}{{\rm Ad}}
\newcommand{\Aut}{{\rm Aut}}
\newcommand{\Cl}{{\rm Cl}}
\newcommand{\Co}{{\rm Co}}
\newcommand{\DES}{{\rm DES}}
\newcommand{\Diff}{{\rm Diff}}
\newcommand{\Dom}{{\rm Dom}}
\newcommand{\Hol}{{\rm Hol}}
\newcommand{\Mon}{{\rm Mon}}
\newcommand{\Hom}{{\rm Hom}}
\newcommand{\Ker}{{\rm Ker}}
\newcommand{\Ind}{{\rm Ind}}
\newcommand{\IM}{{\rm Im}}
\newcommand{\Is}{{\rm Is}}
\newcommand{\ID}{{\rm id}}
\newcommand{\GL}{{\rm GL}}
\newcommand{\Iso}{{\rm Iso}}
\newcommand{\Sem}{{\rm Sem}}
\newcommand{\St}{{\rm St}}
\newcommand{\Sym}{{\rm Sym}}
\newcommand{\SU}{{\rm SU}}
\newcommand{\Tor}{{\rm Tor}}
\newcommand{\U}{{\rm U}}
\newcommand{\A}{\mathcal A}
\newcommand{\Ce}{\mathcal C}
\newcommand{\D}{\mathcal D}
\newcommand{\E}{\mathcal E}
\newcommand{\F}{\mathcal F}
\newcommand{\G}{\mathcal G}
\newcommand{\Q}{\mathcal Q}
\newcommand{\R}{\mathcal R}
\newcommand{\cS}{\mathcal S}
\newcommand{\cU}{\mathcal U}
\newcommand{\W}{\mathcal W}
\newcommand{\bA}{\mathbb{A}}
\newcommand{\bB}{\mathbb{B}}
\newcommand{\bC}{\mathbb{C}}
\newcommand{\bD}{\mathbb{D}}
\newcommand{\bE}{\mathbb{E}}
\newcommand{\bF}{\mathbb{F}}
\newcommand{\bG}{\mathbb{G}}
\newcommand{\bK}{\mathbb{K}}
\newcommand{\bM}{\mathbb{M}}
\newcommand{\bN}{\mathbb{N}}
\newcommand{\bO}{\mathbb{O}}
\newcommand{\bP}{\mathbb{P}}
\newcommand{\bR}{\mathbb{R}}
\newcommand{\bV}{\mathbb{V}}
\newcommand{\bZ}{\mathbb{Z}}
\newcommand{\bfE}{\mathbf{E}}
\newcommand{\bfX}{\mathbf{X}}
\newcommand{\bfY}{\mathbf{Y}}
\newcommand{\bfZ}{\mathbf{Z}}
\renewcommand{\O}{\Omega}
\renewcommand{\o}{\omega}
\newcommand{\vp}{\varphi}
\newcommand{\vep}{\varepsilon}
\newcommand{\diag}{{\rm diag}}
\newcommand{\grp}{{\mathbb G}}
\newcommand{\dgrp}{{\mathbb D}}
\newcommand{\desp}{{\mathbb D^{\rm{es}}}}
\newcommand{\Geod}{{\rm Geod}}
\newcommand{\geod}{{\rm geod}}
\newcommand{\hgr}{{\mathbb H}}
\newcommand{\mgr}{{\mathbb M}}
\newcommand{\ob}{{\rm Ob}}
\newcommand{\obg}{{\rm Ob(\mathbb G)}}
\newcommand{\obgp}{{\rm Ob(\mathbb G')}}
\newcommand{\obh}{{\rm Ob(\mathbb H)}}
\newcommand{\Osmooth}{{\Omega^{\infty}(X,*)}}
\newcommand{\ghomotop}{{\rho_2^{\square}}}
\newcommand{\gcalp}{{\mathbb G(\mathcal P)}}
\newcommand{\rf}{{R_{\mathcal F}}}
\newcommand{\glob}{{\rm glob}}
\newcommand{\loc}{{\rm loc}}
\newcommand{\TOP}{{\rm TOP}}
\newcommand{\wti}{\widetilde}
\newcommand{\what}{\widehat}
\renewcommand{\a}{\alpha}
\newcommand{\be}{\beta}
\newcommand{\ga}{\gamma}
\newcommand{\Ga}{\Gamma}
\newcommand{\de}{\delta}
\newcommand{\del}{\partial}
\newcommand{\ka}{\kappa}
\newcommand{\si}{\sigma}
\newcommand{\ta}{\tau}
\newcommand{\lra}{{\longrightarrow}}
\newcommand{\ra}{{\rightarrow}}
\newcommand{\rat}{{\rightarrowtail}}
\newcommand{\oset}[1]{\overset {#1}{\ra}}
\newcommand{\osetl}[1]{\overset {#1}{\lra}}
\newcommand{\hr}{{\hookrightarrow}}
\begin{document}
\section{An Overview of Algebraic Topology topics}


\subsection{Introduction}
\emph{Algebraic topology} (AT) utilizes algebraic approaches to solve topological problems,
such as the classification of surfaces, proving duality theorems for manifolds and 
approximation theorems for topological spaces. A central problem in algebraic topology 
is to find algebraic invariants of topological spaces, which is usually carried out by means
of homotopy, homology and cohomology groups. There are close connections between algebraic topology, 
\PMlinkname{Algebraic Geometry (AG)}{AlgebraicGeometry}, and Non-commutative Geometry/NAAT. On the other hand, there are also close ties between algebraic geometry and number theory. 


\subsection{Outline}
\begin{enumerate}

\item Homotopy theory and fundamental groups
\item Topology and groupoids; \PMlinkname{van Kampen theorem}{VanKampensTheorem} 
\item Homology and cohomology theories
\item Duality
\item Category theory applications in algebraic topology
\item Index of categories, functors and natural transformations
\item \PMlinkexternal{Grothendieck's Descent theory}{http://www.uclouvain.be/17501.html}
\item `Anabelian geometry'
\item Categorical Galois theory
\item Higher dimensional algebra (HDA)
\item Quantum algebraic topology (QAT)
\item Quantum Geometry
\item Non-Abelian algebraic topology (NAAT)
\end{enumerate}

\subsection{Homotopy theory and fundamental groups}
\begin{enumerate}
\item Homotopy
\item Fundamental group of a space
\item Fundamental theorems
\item van Kampen theorem
\item Whitehead groups, torsion and towers
\item Postnikov towers
\end{enumerate}


\subsection{Topology and Groupoids}
\begin{enumerate}
\item Topology definition, axioms and basic concepts
\item Fundamental groupoid
\item Topological groupoid
\item Classifying space
\item van Kampen theorem for groupoids
\item Groupoid pushout theorem 
\item Double groupoids and crossed modules
\item new4

\end{enumerate}


\subsection{Homology theory}
\begin{enumerate}

\item Homology group
\item Homology sequence
\item Homology complex
\item Homological Algebra

\end{enumerate}


\subsection{Cohomology theory}
\begin{enumerate}

\item Cohomology group
\item Cohomology sequence
\item DeRham cohomology
\item new4

\end{enumerate}

\subsection{Non-Abelian Algebraic Topology}
\begin{enumerate}

\item Crossed Complexes
\item Modules
\item Cross-modules
\item Omega-Groupoids
\item Double Groupoids: Homotopy Double Groupoid of a Hausdorff Space
\item Double Category
\item Groupoid Category
\item Algebroids
\item Higher Homotopy van Kampen Theorem

\end{enumerate}

\emph{...more to come}
%%%%%
%%%%%
\end{document}
