\documentclass[12pt]{article}
\usepackage{pmmeta}
\pmcanonicalname{SardsTheorem}
\pmcreated{2013-03-22 13:04:09}
\pmmodified{2013-03-22 13:04:09}
\pmowner{mathcam}{2727}
\pmmodifier{mathcam}{2727}
\pmtitle{Sard's theorem}
\pmrecord{9}{33477}
\pmprivacy{1}
\pmauthor{mathcam}{2727}
\pmtype{Theorem}
\pmcomment{trigger rebuild}
\pmclassification{msc}{57R35}
\pmrelated{Residual}
\pmrelated{BaireCategoryTheorem}
\pmdefines{critical point}
\pmdefines{critical value}
\pmdefines{regular value}

\endmetadata

\usepackage{amssymb}
\usepackage{amsmath}
\usepackage{amsfonts}
\begin{document}
Let $\phi : X^n \rightarrow Y^m$ be a smooth map on smooth manifolds. A {\it critical point} of $\phi$ is a point $p\in X$ such that the differential $\phi_* : T_pX \rightarrow T_{\phi(p)}Y$ considered as a linear transformation of real vector spaces has \PMlinkname{rank}{RankLinearMapping} $<m$. A {\it critical value} of $\phi$ is the image of a critical point. A {\it regular value} of $\phi$ is a point $q\in Y$ which is not the image of any critical point. In particular, $q$ is a regular value of $\phi$ if $q\in Y \setminus \phi(X)$.

Following Spivak \cite{Spivak}, we say a subset $V$ of $Y^m$ \emph{has measure zero} if there is a sequence of coordinate charts $(x_i,U_i)$ whose union contains $V$ and such that $x_i(U_i\cap V)$ has measure 0 (in the usual sense) in $\mathbb{R}^m$ for all $i$.  With that definition, we can now state:

{\bf Sard's Theorem.} Let $\phi : X \rightarrow Y$ be a smooth map on smooth manifolds. Then the set of critical values of $\phi$ has measure zero.

\begin{thebibliography}{9}
\bibitem[Spivak]{Spivak} Spivak, Michael.  \emph{A Comprehensive Introduction to Differential Geometry.}  Volume I, Third Edition.  Publish of Perish, Inc.  Houston, Texas.  1999.
\end{thebibliography}
%%%%%
%%%%%
\end{document}
