\documentclass[12pt]{article}
\usepackage{pmmeta}
\pmcanonicalname{Embedding}
\pmcreated{2013-03-22 14:52:46}
\pmmodified{2013-03-22 14:52:46}
\pmowner{CWoo}{3771}
\pmmodifier{CWoo}{3771}
\pmtitle{embedding}
\pmrecord{8}{36557}
\pmprivacy{1}
\pmauthor{CWoo}{3771}
\pmtype{Definition}
\pmcomment{trigger rebuild}
\pmclassification{msc}{57R40}
\pmsynonym{differential embedding}{Embedding}
\pmdefines{Whitney's theorem}

\endmetadata

% this is the default PlanetMath preamble.  as your knowledge
% of TeX increases, you will probably want to edit this, but
% it should be fine as is for beginners.

% almost certainly you want these
\usepackage{amssymb,amscd}
\usepackage{amsmath}
\usepackage{amsfonts}

% used for TeXing text within eps files
%\usepackage{psfrag}
% need this for including graphics (\includegraphics)
%\usepackage{graphicx}
% for neatly defining theorems and propositions
%\usepackage{amsthm}
% making logically defined graphics
%%%\usepackage{xypic}

% there are many more packages, add them here as you need them

% define commands here
\begin{document}
\PMlinkescapeword{characterization}
\PMlinkescapeword{states}

Let $M$ and $N$ be manifolds and $f\colon M\to N$ a smooth map.  Then $f$ is an \emph{embedding} if 
\begin{enumerate}
\item $f(M)$ is a submanifold of $N$, and 
\item $f\colon M\to f(M)$ is a diffeomorphism.  (There's an abuse of notation here.  This should really be restated as the map $g\colon M\to f(M)$ defined by $g(p)=f(p)$ is a diffeomorphism.)
\end{enumerate}

The above characterization can be equivalently stated:
$f\colon M\to N$ is an embedding if
\begin{enumerate}
\item $f$ is an immersion, and 
\item by abuse of notation, $f\colon M\to f(M)$ is a homeomorphism.
\end{enumerate}

\textbf{Remark}.  A celebrated theorem of Whitney states that every $n$ dimensional manifold admits an embedding into $\mathbb{R}^{2n+1}$.
%%%%%
%%%%%
\end{document}
