\documentclass[12pt]{article}
\usepackage{pmmeta}
\pmcanonicalname{mathcalCrTopologies}
\pmcreated{2013-03-22 14:08:27}
\pmmodified{2013-03-22 14:08:27}
\pmowner{Koro}{127}
\pmmodifier{Koro}{127}
\pmtitle{$\mathcal{C}^r$ topologies}
\pmrecord{5}{35555}
\pmprivacy{1}
\pmauthor{Koro}{127}
\pmtype{Definition}
\pmcomment{trigger rebuild}
\pmclassification{msc}{57R12}
\pmsynonym{Whitney topology}{mathcalCrTopologies}
\pmsynonym{compact-open $\mathcal{C}^r$ topology}{mathcalCrTopologies}
\pmsynonym{weak $\mathcal{C}^r$ topology}{mathcalCrTopologies}
\pmsynonym{strong $\mathcak{C}^r$ topology}{mathcalCrTopologies}
\pmrelated{ConjectureApproximationTheoremHoldsForWhitneyCrMNSpaces}
\pmrelated{ApproximationTheoremAppliedToWhitneyCrMNSpaces}

% this is the default PlanetMath preamble.  as your knowledge
% of TeX increases, you will probably want to edit this, but
% it should be fine as is for beginners.

% almost certainly you want these
\usepackage{amssymb}
\usepackage{amsmath}
\usepackage{amsfonts}
\usepackage{mathrsfs}

% used for TeXing text within eps files
%\usepackage{psfrag}
% need this for including graphics (\includegraphics)
%\usepackage{graphicx}
% for neatly defining theorems and propositions
%\usepackage{amsthm}
% making logically defined graphics
%%%\usepackage{xypic}

% there are many more packages, add them here as you need them

% define commands here
\newcommand{\C}{\mathcal{C}}
\newcommand{\R}{\mathbb{R}}
\newcommand{\N}{\mathbb{N}}
\newcommand{\Z}{\mathbb{Z}}
\newcommand{\Per}{\operatorname{Per}}
\begin{document}
The $\C^r$ Whitney (or strong) topology is a topology
assigned to the space $\C^r(M,N)$ of mappings from
a $\C^r$ manifold $M$ to a $\C^r$ manifold $N$ having
$r$ continuous derivatives . It gives a notion of proximity
of two $\C^r$ mappings, and it allows us to speak of ``robustness''
of properties of a mapping. For example, the
property of being an embedding is robust: if $f\colon M\to N$
is a $\C^r$ embedding, then there is a strong $\C^r$
neighborhood of $f$ in which any $\C^r$ mapping $g\colon M\to N$
is an embedding.

Given a locally finite atlas $\{(U_i, \phi_i):i\in I\}$ and compact sets
$K_i\subset U_i$ such that there are charts
$\{(V_i,\psi_i) : i\in I\}$ of $N$ for which
$f(K_i)\subset V_i$ for all $i\in I$, and given a sequence
$\{\epsilon_i>0 : i\in I\}$, we define the basic neighborhood
$$\mathcal{U}^r\left(f,\phi,\psi,\{K_i:i\in I\},\{\epsilon_i:i\in I\}\right)$$
as the set of $C^r$ mappings $g\colon M\to N$ such that for all $i\in I$
we have $g(K_i)\subset V_i$ and
$$\sup_{x\in \phi_i(K_i), 0\leq k\leq r} 
||D^k(\psi_if\phi_i^{-1})(x) - D^k(\psi_ig\phi_i^{-1})(x)|| <\epsilon_i.$$
That is, those maps $g$ that are close to $f$ and have their first $r$
derivatives close to the respective first $r$-th
derivatives of $f$, in local coordinates.
It can be checked that the set of all such neighborhoods forms a basis
for a topology, which we call the Whitney or strong $\C^r$ 
topology of $\C^r(M,N)$.

The weak $\C^r$ topology, or $\C^r$ compact-open topology, is defined
in the same fashion but instead of choosing
$\{(U_i,\phi_i):i\in I\}$ to be a locally finite atlas for $M$,
we require it to be an arbitrary \emph{finite} family of charts
(possibly not covering $M$).

The space $\C^r(M,N)$ with the weak or strong topologies is denoted by
$\C^r_W(M,N)$ and $\C^r_S(M,N)$, respectively.

We have that $\C^r_W(M,N)$ is always metrizable (with a complete metric)
and separable. On the other hand, $\C^r_S(M,N)$ is not even first countable (thus, not metrizable) when $M$ is not compact; however, it is a Baire space. When $M$ is compact, the weak and strong topologies coincide.
%%%%%
%%%%%
\end{document}
