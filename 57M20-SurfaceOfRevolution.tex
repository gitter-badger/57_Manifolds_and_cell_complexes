\documentclass[12pt]{article}
\usepackage{pmmeta}
\pmcanonicalname{SurfaceOfRevolution}
\pmcreated{2013-03-22 17:17:08}
\pmmodified{2013-03-22 17:17:08}
\pmowner{pahio}{2872}
\pmmodifier{pahio}{2872}
\pmtitle{surface of revolution}
\pmrecord{14}{39627}
\pmprivacy{1}
\pmauthor{pahio}{2872}
\pmtype{Topic}
\pmcomment{trigger rebuild}
\pmclassification{msc}{57M20}
\pmclassification{msc}{51M04}
%\pmkeywords{rotation}
\pmrelated{SurfaceOfRevolution}
\pmrelated{PappussTheoremForSurfacesOfRevolution}
\pmrelated{QuadraticSurfaces}
\pmrelated{ConicalSurface}
\pmrelated{Torus}
\pmrelated{SolidOfRevolution}
\pmrelated{LeastSurfaceOfRevolution}
\pmrelated{ConeInMathbbR3}
\pmdefines{surface of revolution}
\pmdefines{axis of revolution}
\pmdefines{circle of latitude}
\pmdefines{meridian curve}
\pmdefines{0-meridian}
\pmdefines{cone of revolution}
\pmdefines{asymptote cone}
\pmdefines{catenoid}

\endmetadata

% this is the default PlanetMath preamble.  as your knowledge
% of TeX increases, you will probably want to edit this, but
% it should be fine as is for beginners.

% almost certainly you want these
\usepackage{amssymb}
\usepackage{amsmath}
\usepackage{amsfonts}

% used for TeXing text within eps files
%\usepackage{psfrag}
% need this for including graphics (\includegraphics)
%\usepackage{graphicx}
% for neatly defining theorems and propositions
 \usepackage{amsthm}
% making logically defined graphics
%%%\usepackage{xypic}

% there are many more packages, add them here as you need them

% define commands here

\theoremstyle{definition}
\newtheorem*{thmplain}{Theorem}

\begin{document}
If a curve in $\mathbb{R}^3$ rotates about a line, it generates a {\em surface of revolution}.  The line is called the {\em axis of revolution}.\, Every point of the curve generates a {\em circle of latitude}.  If the surface is intersected by a half-plane beginning from the axis of revolution, the intersection curve is a {\em meridian curve}.  One can always think that the surface of revolution is generated by the rotation of a certain meridian, which may be called the {\em 0-meridian}.

Let\, $y = f(x)$\, be a curve of the $xy$-plane rotating about the $x$-axis.  Then any point \,$(x,\,y)$\, of this 0-meridian draws a circle of latitude, parallel to the $yz$-plane, with centre on the $x$-axis and with the radius $|f(x)|$.  So the $y$- and $z$-coordinates of each point on this circle satisfy the equation
                    $$y^2\!+\!z^2 \;=\; [f(x)]^2.$$
This equation is thus satisfied by all points\, $(x,\,y,\,z)$\, of the surface of revolution and therefore it is the equation of the whole surface of revolution.

More generally, if the equation of the meridian curve in the $xy$-plane is given in the implicit form 
\,$F(x,\,y) = 0$,\, then the equation of the surface of revolution may be written
$$F(x,\,\sqrt{y^2\!+\!z^2}) \;=\; 0.$$

\textbf{Examples.}

When the catenary \,$y = a\cosh\frac{x}{a}$\, rotates about the $x$-axis, it generates the {\em catenoid}
$$y^2\!+\!z^2 \;=\; a^2\cosh^2\frac{x}{a}.$$
The catenoid is the only surface of revolution being also a minimal surface.

The quadratic surfaces of revolution:
\begin{itemize}
\item When the ellipse \,$\displaystyle\frac{x^2}{a^2}+\frac{y^2}{b^2} = 1$\, rotates about the $x$-axis, we get the ellipsoid
$$\frac{x^2}{a^2}+\frac{y^2\!+\!z^2}{b^2} \;=\; 1.$$
This is a {\em stretched ellipsoid}, if\, $a > b$,\, and a {\em flattened ellipsoid}, if\, $a < b$, and a sphere of radius $a$, if\, $a = b$.
\item When the parabola \,$y^2 = 2px$ (with $p$ the {\em latus rectum} or the parameter of parabola) rotates about the $x$-axis, we get the {\em paraboloid of revolution}
$$y^2\!+\!z^2 \;=\; 2px.$$
\item When we let the conjugate hyperbolas and their common asymptotes 
\,$\displaystyle\frac{x^2}{a^2}-\frac{y^2}{b^2} = s$\, (with\, $s = 1,\,-1,\,0$) rotate about the $x$-axis, we obtain the {\em two-sheeted hyperboloid}
$$\frac{x^2}{a^2}-\frac{y^2\!+\!z^2}{b^2} \;=\; 1,$$
the {\em one-sheeted hyperboloid}
$$\frac{x^2}{a^2}-\frac{y^2\!+\!z^2}{b^2} \;=\; -1$$
and the {\em cone of revolution}
$$\frac{x^2}{a^2}-\frac{y^2\!+\!z^2}{b^2} \;=\; 0,$$ 
which apparently is the common {\em asymptote cone} of both hyperboloids.
\end{itemize}


\begin{thebibliography}{8}
\bibitem{LP}{\sc Lauri Pimi\"a}: {\em Analyyttinen geometria}.\, Werner S\"oderstr\"om Osakeyhti\"o, Porvoo and Helsinki (1958).
\end{thebibliography} 

%%%%%
%%%%%
\end{document}
