\documentclass[12pt]{article}
\usepackage{pmmeta}
\pmcanonicalname{CrossingChange}
\pmcreated{2013-03-22 16:29:14}
\pmmodified{2013-03-22 16:29:14}
\pmowner{PrimeFan}{13766}
\pmmodifier{PrimeFan}{13766}
\pmtitle{crossing change}
\pmrecord{6}{38656}
\pmprivacy{1}
\pmauthor{PrimeFan}{13766}
\pmtype{Definition}
\pmcomment{trigger rebuild}
\pmclassification{msc}{57M25}

% this is the default PlanetMath preamble.  as your knowledge
% of TeX increases, you will probably want to edit this, but
% it should be fine as is for beginners.

% almost certainly you want these
\usepackage{amssymb}
\usepackage{amsmath}
\usepackage{amsfonts}

% used for TeXing text within eps files
%\usepackage{psfrag}

% need this for including graphics (\includegraphics)
\usepackage{graphicx}

% for neatly defining theorems and propositions
%\usepackage{amsthm}
% making logically defined graphics
%%%\usepackage{xypic}

% there are many more packages, add them here as you need them

% define commands here

\begin{document}
In a knot, a {\em crossing change} occurs when a positive crossing (or crossing over) is followed by a negative crossing (or crossing under). An alternating knot, for example, has exactly as many crossing changes as it has crossings.

For the purpose of illustrating crossing changes, a non-alternating knot is more interesting. In the following diagram, if we start at the point indicated by the blue dot and proceed clockwise, the red arrows indicate crossing changes while the green arrows indicate a crossing the same as the previous crossing. One crossing change after that and there are no further arrows in the diagram until the blue dot.

\begin{center}
\includegraphics{CrossingChangeDiagram}
\end{center}
%%%%%
%%%%%
\end{document}
