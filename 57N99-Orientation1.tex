\documentclass[12pt]{article}
\usepackage{pmmeta}
\pmcanonicalname{Orientation1}
\pmcreated{2013-03-22 12:56:24}
\pmmodified{2013-03-22 12:56:24}
\pmowner{PrimeFan}{13766}
\pmmodifier{PrimeFan}{13766}
\pmtitle{orientation}
\pmrecord{15}{33297}
\pmprivacy{1}
\pmauthor{PrimeFan}{13766}
\pmtype{Definition}
\pmcomment{trigger rebuild}
\pmclassification{msc}{57N99}
\pmrelated{ThomClass}
\pmdefines{orientable}
\pmdefines{oriented}
\pmdefines{orientable manifold}
\pmdefines{oriented manifold}
\pmdefines{fundamental class}
\pmdefines{orientation class}
\pmdefines{local orientation}

%\documentclass{amsart}
\usepackage{amsmath}
%\usepackage[all,poly,knot,dvips]{xy}
%\usepackage{pstricks,pst-poly,pst-node,pstcol}


\usepackage{amssymb,latexsym}

\usepackage{amsthm,latexsym}
\usepackage{eucal,latexsym}

% THEOREM Environments --------------------------------------------------

\newtheorem{thm}{Theorem}
 \newtheorem*{mainthm}{Main~Theorem}
 \newtheorem{cor}[thm]{Corollary}
 \newtheorem{lem}[thm]{Lemma}
 \newtheorem{prop}[thm]{Proposition}
 \newtheorem{claim}[thm]{Claim}
 \theoremstyle{definition}
 \newtheorem{defn}[thm]{Definition}
 \theoremstyle{remark}
 \newtheorem{rem}[thm]{Remark}
 \numberwithin{equation}{subsection}


%---------------------  Greek letters, etc ------------------------- 

\newcommand{\CA}{\mathcal{A}}
\newcommand{\CC}{\mathcal{C}}
\newcommand{\CM}{\mathcal{M}}
\newcommand{\CP}{\mathcal{P}}
\newcommand{\CS}{\mathcal{S}}
\newcommand{\BC}{\mathbb{C}}
\newcommand{\BN}{\mathbb{N}}
\newcommand{\BR}{\mathbb{R}}
\newcommand{\BZ}{\mathbb{Z}}
\newcommand{\FF}{\mathfrak{F}}
\newcommand{\FL}{\mathfrak{L}}
\newcommand{\FM}{\mathfrak{M}}
\newcommand{\Ga}{\alpha}
\newcommand{\Gb}{\beta}
\newcommand{\Gg}{\gamma}
\newcommand{\GG}{\Gamma}
\newcommand{\Gd}{\delta}
\newcommand{\GD}{\Delta}
\newcommand{\Ge}{\varepsilon}
\newcommand{\Gz}{\zeta}
\newcommand{\Gh}{\eta}
\newcommand{\Gq}{\theta}
\newcommand{\GQ}{\Theta}
\newcommand{\Gi}{\iota}
\newcommand{\Gk}{\kappa}
\newcommand{\Gl}{\lambda}
\newcommand{\GL}{\Lamda}
\newcommand{\Gm}{\mu}
\newcommand{\Gn}{\nu}
\newcommand{\Gx}{\xi}
\newcommand{\GX}{\Xi}
\newcommand{\Gp}{\pi}
\newcommand{\GP}{\Pi}
\newcommand{\Gr}{\rho}
\newcommand{\Gs}{\sigma}
\newcommand{\GS}{\Sigma}
\newcommand{\Gt}{\tau}
\newcommand{\Gu}{\upsilon}
\newcommand{\GU}{\Upsilon}
\newcommand{\Gf}{\varphi}
\newcommand{\GF}{\Phi}
\newcommand{\Gc}{\chi}
\newcommand{\Gy}{\psi}
\newcommand{\GY}{\Psi}
\newcommand{\Gw}{\omega}
\newcommand{\GW}{\Omega}
\newcommand{\Gee}{\epsilon}
\newcommand{\Gpp}{\varpi}
\newcommand{\Grr}{\varrho}
\newcommand{\Gff}{\phi}
\newcommand{\Gss}{\varsigma}
\newcommand{\n}{\newline}
\def\co{\colon\thinspace}
\begin{document}
There are many definitions of an orientation of a manifold. The most
general, in the sense that it doesn't require any extra
\PMlinkescapetext{structure} on the manifold, is based on
(co-)homology theory. For this article manifold means a connected,
{\em topological} manifold possibly with boundary.
\begin{thm} Let $M$ be a closed, $n$--dimensional
  \emph{manifold}. Then $H_n(M\,;\mathbb{Z})$ the top dimensional
  homology group of $M$, is either trivial ($\{0\}$) or isomorphic
  to $ \mathbb{Z}$.
\end{thm}

\begin{defn} A closed  $n$--manifold is called \emph{orientable} if its top
 homology group is isomorphic to the integers.
  An \emph{orientation} of  $M$ is a choice of a particular isomorphism
   $$\mathfrak{o}\co \mathbb{Z} \to H_n(M\,;\mathbb{Z}).$$
An oriented manifold is a (necessarily orientable) manifold $M$ endowed with 
an orientation. 
If $(M,\mathfrak{o})$ is an oriented manifold then  $\mathfrak{o}(1)$ is called
the \emph{fundamental class} of $M$ , or the \emph{orientation class} of $M$, and is denoted
 by $[M]$.
\end{defn}

\begin{rem} Notice that since $\mathbb{Z}$ has exactly two
  automorphisms an orientable manifold admits two possible
  orientations.
\end{rem}

\begin{rem} The above definition could be given using cohomology instead of homology. 
\end{rem}

The top dimensional homology of a non-closed manifold is always
trivial, so it is trickier to define orientation for those
beasts. One approach (which we will not follow) is to use special
kind of homology (for example relative to the boundary for compact
manifolds with boundary). The approach we follow defines (global)
orientation as compatible fitting together of local orientations. We
start with manifolds without boundary.

\begin{thm}Let $M$ be an $n$-manifold without boundary and $x\in
  M$. Then the relative homology group
          $$H_n(M,M\setminus x\,;\mathbb{Z}) \cong \mathbb{Z}$$
\end{thm}
\begin{defn} Let $M$ be an $n$-manifold and $x\in M$. An  {\em orientation 
of} $M$  {\em at} $x$  is a choice of an isomorphism 
 $$\mathfrak{o}_x \co \mathbb{Z} \to H_n(M,M\setminus x\,;\mathbb{Z}).$$
  \end{defn}

  \PMlinkescapetext{One way} to make precise the notion of nicely
  fitting together of orientations at points, is to require that for
  nearby points the orientations are defined in a
  \PMlinkescapetext{uniform} way.
  \begin{thm}
    Let $U$ be an open subset of $M$ that is homeomorphic to $\BR^n$
    (e.g. the domain of a chart).  Then,
 $$H_n(M,M\setminus U \,; \BZ)\cong \BZ.$$
\end{thm}

\begin{defn} Let $U$ be an open subset of $M$ that is homeomorphic
  to $\BR^n$. A \emph{local orientation} of $M$ on $U$ is a choice
  of an isomorphism
 $$\mathfrak{o}_U \co H_n(M,M\setminus U \,; \BZ) \to \BZ.$$
\end{defn}

Now notice that with $U$ as above and $x\in U$ the inclusion
  $$\imath^U_x\co M\setminus U  \hookrightarrow  M\setminus x$$
\PMlinkescapetext{induces} a map (actually isomorphism) 
      $$\imath^U_{x\,\,*}\co H_n(M,M\setminus U \,; \BZ) \to H_n(M,M\setminus x\,;\mathbb{Z})$$
      and therefore a local orientation at $U$
      \PMlinkescapetext{induces} (by composing with the above
      isomorphism) an orientation at each point $x\in U$.  It is
      \PMlinkescapetext{natural} to declare that all these
      orientations fit nicely together.



\begin{defn}
  Let $M$ be a manifold with non-empty boundary, $\partial M\neq
  \emptyset$. $M$ is called \emph{orientable} if its double
     $$\hat{M}:=M\bigcup_{\partial M}M$$
is orientable, where $\bigcup_{\partial M}$ denotes gluing along the boundary.\n
An orientation of $M$ is determined by an orientation of $\hat{M}$.    
\end{defn}
%%%%%
%%%%%
\end{document}
