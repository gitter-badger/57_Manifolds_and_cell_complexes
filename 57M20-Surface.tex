\documentclass[12pt]{article}
\usepackage{pmmeta}
\pmcanonicalname{Surface}
\pmcreated{2013-03-22 16:01:43}
\pmmodified{2013-03-22 16:01:43}
\pmowner{juanman}{12619}
\pmmodifier{juanman}{12619}
\pmtitle{surface}
\pmrecord{13}{38071}
\pmprivacy{1}
\pmauthor{juanman}{12619}
\pmtype{Definition}
\pmcomment{trigger rebuild}
\pmclassification{msc}{57M20}
\pmrelated{Manifold}
\pmrelated{NonOrientableSurface}

\endmetadata

% this is the default PlanetMath preamble.  as your knowledge
% of TeX increases, you will probably want to edit this, but
% it should be fine as is for beginners.

% almost certainly you want these
\usepackage{amssymb}
\usepackage{amsmath}
\usepackage{amsfonts}

% used for TeXing text within eps files
%\usepackage{psfrag}
% need this for including graphics (\includegraphics)
%\usepackage{graphicx}
% for neatly defining theorems and propositions
%\usepackage{amsthm}
% making logically defined graphics
%%%\usepackage{xypic}

% there are many more packages, add them here as you need them

% define commands here

\begin{document}
A \emph{surface} is a two-dimensional topological manifold.
A closed surface is a surface without boundary.

A result called the ``classification theorem'' gives us a symbolic semantics, matching the geometrical view point, in terms of genera, orientability and number of boundary components. Together with the connected sum operation, they make available a powerful language to be explored and exploited.

As an example of a surface take $T=S^1\times S^1$ the two torus, the boundary of a solid sugar donut shaped cake $D^2\times S^1$, where $S^1$ is the familiar modulus one complex numbers.

%%%%%
%%%%%
\end{document}
