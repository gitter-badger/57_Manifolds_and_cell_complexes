\documentclass[12pt]{article}
\usepackage{pmmeta}
\pmcanonicalname{ExoticR4s}
\pmcreated{2013-03-22 15:37:33}
\pmmodified{2013-03-22 15:37:33}
\pmowner{whm22}{2009}
\pmmodifier{whm22}{2009}
\pmtitle{exotic R4's}
\pmrecord{21}{37551}
\pmprivacy{1}
\pmauthor{whm22}{2009}
\pmtype{Definition}
\pmcomment{trigger rebuild}
\pmclassification{msc}{57R12}
\pmclassification{msc}{14J80}
%\pmkeywords{Donaldson's Theorem}
\pmrelated{DonaldsonsTheorem}
\pmrelated{DonaldsonFreedmanExoticR4}

\endmetadata

% this is the default PlanetMath preamble.  as your knowledge
% of TeX increases, you will probably want to edit this, but
% it should be fine as is for beginners.

% almost certainly you want these
\usepackage{amssymb}
\usepackage{amsmath}
\usepackage{amsfonts}

% used for TeXing text within eps files
%\usepackage{psfrag}
% need this for including graphics (\includegraphics)
%\usepackage{graphicx}
% for neatly defining theorems and propositions
%\usepackage{amsthm}
% making logically defined graphics
%%%\usepackage{xypic}

% there are many more packages, add them here as you need them

% define commands here
\newcommand {\R} {\mathbb R}
\begin{document}
If $n \neq 4$ then the smooth manifolds homeomorphic to a given topological $n$- manifold, $M$, are parameterized by some discrete algebraic invariant of $M$.  In particular there is a unique smooth manifold homeomorphic to $\R^n$.

\bigskip
By contrast one may choose uncountably many open sets in $\R^4$, which are all homeomorphic to $\R^4$, but which are pairwise non-diffeomorphic.

\bigskip
A smooth manifold homeomorphic to $\R^4$, but not diffeomorphic to it is called an \emph{exotic} $\R^4$.

\bigskip
Given an exotic $\R^4$, $E$, we have a diffeomorphism $E \times \R \to \R^5$.  (As there is only one smooth manifold homeomorphic to $\R^5$).  Hence exotic  $\R^4$'s may be identified with closed submanifolds of $\R^5$.  In particular this means the cardinality of the set of exotic $\R^4$'s is precisely continuum.

\bigskip
Historically, Donaldson's theorem led to the discovery of the Donaldson Freedman exotic $\R^4$.
%%%%%
%%%%%
\end{document}
