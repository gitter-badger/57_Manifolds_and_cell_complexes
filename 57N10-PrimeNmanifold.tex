\documentclass[12pt]{article}
\usepackage{pmmeta}
\pmcanonicalname{PrimeNmanifold}
\pmcreated{2013-03-22 16:05:37}
\pmmodified{2013-03-22 16:05:37}
\pmowner{juanman}{12619}
\pmmodifier{juanman}{12619}
\pmtitle{prime n-manifold}
\pmrecord{14}{38155}
\pmprivacy{1}
\pmauthor{juanman}{12619}
\pmtype{Feature}
\pmcomment{trigger rebuild}
\pmclassification{msc}{57N10}

\endmetadata

% this is the default PlanetMath preamble.  as your knowledge
% of TeX increases, you will probably want to edit this, but
% it should be fine as is for beginners.

% almost certainly you want these
\usepackage{amssymb}
\usepackage{amsmath}
\usepackage{amsfonts}

% used for TeXing text within eps files
%\usepackage{psfrag}
% need this for including graphics (\includegraphics)
%\usepackage{graphicx}
% for neatly defining theorems and propositions
%\usepackage{amsthm}
% making logically defined graphics
%%%\usepackage{xypic}

% there are many more packages, add them here as you need them

% define commands here

\begin{document}
A \PMlinkname{n-manifold}{TopologicalManifold} $M$ is called \PMlinkescapetext{\emph{prime}} if for all the factorizations of $M$, as a \PMlinkname{connected sum}{ConnectedSum2} $M=M_1\#M_2$, one finds that one of the factors $M_1$ or $M_2$ is the n-sphere $S^n$. 

The importance for 3-manifold is the existence and uniqueness prime decomposition theorem which says: Each 3-manifold can be decomposed as a connected sum of prime manifolds. 
%%%%%
%%%%%
\end{document}
