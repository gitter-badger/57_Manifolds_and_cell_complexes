\documentclass[12pt]{article}
\usepackage{pmmeta}
\pmcanonicalname{HairyBallTheorem}
\pmcreated{2013-03-22 13:11:33}
\pmmodified{2013-03-22 13:11:33}
\pmowner{rspuzio}{6075}
\pmmodifier{rspuzio}{6075}
\pmtitle{hairy ball theorem}
\pmrecord{12}{33646}
\pmprivacy{1}
\pmauthor{rspuzio}{6075}
\pmtype{Theorem}
\pmcomment{trigger rebuild}
\pmclassification{msc}{57R22}
\pmsynonym{porcupine theorem}{HairyBallTheorem}
\pmsynonym{Poincaré-Hopf theorem}{HairyBallTheorem}
\pmdefines{Poincar\'e-Hopf index theorem}

\endmetadata

% this is the default PlanetMath preamble.  as your knowledge
% of TeX increases, you will probably want to edit this, but
% it should be fine as is for beginners.

% almost certainly you want these
\usepackage{amssymb}
\usepackage{amsmath}
\usepackage{amsfonts}

% used for TeXing text within eps files
%\usepackage{psfrag}
% need this for including graphics (\includegraphics)
%\usepackage{graphicx}
% for neatly defining theorems and propositions
\usepackage{amsthm}
% making logically defined graphics
%%%\usepackage{xypic}

% there are many more packages, add them here as you need them

% define commands here
\newtheorem*{theorem*}{Theorem}
\begin{document}
\begin{theorem*}
If $X$ is a vector field on $S^{2n}$, then $X$ has a zero.
Alternatively, there are no continuous unit vector field on
the sphere.  Moreover, the tangent bundle of the sphere is
nontrivial as a bundle, that is, it is not simply a product.
\end{theorem*}

There are two proofs for this.  The first proof is based
on the fact that the antipodal map on $S^{2n}$ is not homotopic to the 
identity map.  The second proof gives the \PMlinkescapetext{hairy ball theorem}
as a corollary of the Poincar\'e-Hopf index theorem.

Near a zero of a vector field, we can consider a small sphere around the zero, and restrict the vector field to that. By normalizing, we get a map from the sphere to itself. We define the index of the vector field at a zero to be the degree of that map.

\begin{theorem*}[Poincar\'e-Hopf index theorem]
If $X$ is a vector field on a compact manifold $M$ with 
isolated zeroes, then $\chi(M)=\sum_{v\in Z(X)}\iota(v)$ 
where $Z(X)$ is the set of zeroes of $X$, and $\iota(v)$ is 
the index of $x$ at $v$, and $\chi(M)$ is the Euler characteristic of $M$.
\end{theorem*}

It is not difficult to show that $S^{2n+1}$ has non-vanishing vector 
fields for all $n$.  A much harder result of Adams shows that the 
tangent bundle of $S^m$ is trivial if and only if $n=0,1,3,7$, 
corresponding to the unit spheres in the 4 real division algebras.

\begin{proof}
First, the low tech proof.  Assume that $S^{2n}$ has a unit vector field
$X$.  Then the \PMlinkname{antipodal map is homotopic to the identity}{AntipodalMapOnSnIsHomotopicToTheIdentityIfAndOnlyIfNIsOdd}.  
But this cannot be, since the degree of the antipodal map is $-1$ and
the degree of the identity map is $+1$.  We therefore reject the
assumption that $X$ is a unit vector field.

This also implies that the tangent bundle of $S^{2n}$ is non-trivial, 
since any trivial bundle has a non-zero section.
\end{proof}

\begin{proof}
Now for the sledgehammer proof.
Suppose $X$ is a nonvanishing vector field on $S^{2n}$.  
Then by the Poincar\'e-Hopf index theorem, the Euler characteristic 
of $S^{2n}$ is $\chi(X)=\sum_{v\in X^{-1}(0)}\iota(v)=0$.  But the Euler characteristic of $S^{2k}$ is $2$.  Hence $X$ must have a zero.
\end{proof}
%%%%%
%%%%%
\end{document}
