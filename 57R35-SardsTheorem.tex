\documentclass[12pt]{article}
\usepackage{pmmeta}
\pmcanonicalname{SardsTheorem}
\pmcreated{2013-03-22 16:12:33}
\pmmodified{2013-03-22 16:12:33}
\pmowner{Mathprof}{13753}
\pmmodifier{Mathprof}{13753}
\pmtitle{Sard's theorem}
\pmrecord{7}{38305}
\pmprivacy{1}
\pmauthor{Mathprof}{13753}
\pmtype{Theorem}
\pmcomment{trigger rebuild}
\pmclassification{msc}{57R35}

% this is the default PlanetMath preamble.  as your knowledge
% of TeX increases, you will probably want to edit this, but
% it should be fine as is for beginners.

% almost certainly you want these
\usepackage{amssymb}
\usepackage{amsmath}
\usepackage{amsfonts}

% used for TeXing text within eps files
%\usepackage{psfrag}
% need this for including graphics (\includegraphics)
%\usepackage{graphicx}
% for neatly defining theorems and propositions
\usepackage{amsthm}
% making logically defined graphics
%%%\usepackage{xypic}

% there are many more packages, add them here as you need them

% define commands here
\newtheorem*{thm}{Theorem}

\begin{document}
\begin{thm} Let $U$ be an open set in $R^m$,  $V$ be an open set in $R^n$ and
$f: U \to V$ be a $C^k$ function. If $k \geq m/n$,  then the set of
critical values of $f$ has $n$-dimensional measure zero.
\end{thm}


\begin{thebibliography}{99}
\bibitem{kr} S.G. Krantz, H. R. Parks, \emph{The Implicit Function Theorem: 
History, Theory, and Applications}, Birkh\"auser, Boston, c. 2002
\bibitem{fe} H. Federer, \emph{Geometric Measure Theory}, Springer-Verlag,
New York, 1969
\end{thebibliography}

%%%%%
%%%%%
\end{document}
