\documentclass[12pt]{article}
\usepackage{pmmeta}
\pmcanonicalname{RoundFunction}
\pmcreated{2013-03-22 15:44:12}
\pmmodified{2013-03-22 15:44:12}
\pmowner{juanman}{12619}
\pmmodifier{juanman}{12619}
\pmtitle{round function}
\pmrecord{11}{37687}
\pmprivacy{1}
\pmauthor{juanman}{12619}
\pmtype{Definition}
\pmcomment{trigger rebuild}
\pmclassification{msc}{57R70}
\pmsynonym{functions with critical loops}{RoundFunction}
%\pmkeywords{Functions on manifolds}
%\pmkeywords{critical points}
\pmrelated{DifferntiableFunction}

\endmetadata

% this is the default PlanetMath preamble.  as your knowledge
% of TeX increases, you will probably want to edit this, but
% it should be fine as is for beginners.

% almost certainly you want these
\usepackage{amssymb}
\usepackage{amsmath}
\usepackage{amsfonts}

% used for TeXing text within eps files
%\usepackage{psfrag}
% need this for including graphics (\includegraphics)
%\usepackage{graphicx}
% for neatly defining theorems and propositions
%\usepackage{amsthm}
% making logically defined graphics
%%%\usepackage{xypic}

% there are many more packages, add them here as you need them

% define commands here
\newcommand{\paren}[1]{\left(\begin{array}{c} #1 \end{array}\right) }
\begin{document}
Let $M$ be a manifold. By a \emph{round function} we \PMlinkescapetext{mean} a function $M\to{\mathbb{R}}$ whose critical points form connected components, each of which is homeomorphic to the circle $S^1$.

For example, let $M$ be the torus. Let $K=]0,2\pi[\times]0,2\pi[$.  Then we know that a map 
$X\colon K\to{\mathbb{R}}^3$ given by 
$$X(\theta,\phi)=((2+\cos\theta)\cos\phi,(2+\cos\theta)\sin\phi,\sin\theta)$$
is a parametrization for almost all of $M$. Now, via the projection $\pi_3\colon{\mathbb{R}}^3\to{\mathbb{R}}$
we get the restriction $G=\pi_3|_M\colon M\to{\mathbb{R}}$ whose critical sets are determined by 
$$\nabla G(\theta,\phi)=\paren{{\partial G\over \partial\theta},{\partial G\over \partial\phi}}(\theta,\phi)=(0,0)$$ 
if and only if $\theta={\pi\over 2},\ {3\pi\over 2}$.

These two values for $\theta$ give the critical set
$$X\paren{{\pi\over 2},\phi}=(2\cos\phi,2\sin\phi,1)$$
$$X\paren{{3\pi\over 2},\phi}=(2\cos\phi,2\sin\phi,-1)$$
which represent two extremal circles over the torus $M$.

Observe that the Hessian for this function  is
$d^2(G)=
\left(\begin{array}{cc}
-\sin\theta & 0 \\
0          & 0 
\end{array}\right)
$
which clearly it reveals itself as of ${\rm rank}(d^2(G))=1$ at the tagged circles, 
making the critical point degenerate, that is, showing that the critical points are not isolated.
%%%%%
%%%%%
\end{document}
