\documentclass[12pt]{article}
\usepackage{pmmeta}
\pmcanonicalname{Orbifold}
\pmcreated{2013-03-22 15:40:06}
\pmmodified{2013-03-22 15:40:06}
\pmowner{guffin}{12505}
\pmmodifier{guffin}{12505}
\pmtitle{orbifold}
\pmrecord{8}{37607}
\pmprivacy{1}
\pmauthor{guffin}{12505}
\pmtype{Definition}
\pmcomment{trigger rebuild}
\pmclassification{msc}{57M07}
\pmsynonym{orbifold structure}{Orbifold}
\pmdefines{orbifold structure}

\endmetadata

% this is the default PlanetMath preamble.  as your knowledge
% of TeX increases, you will probably want to edit this, but
% it should be fine as is for beginners.

% almost certainly you want these
\usepackage{amssymb}
\usepackage{amsmath}
\usepackage{amsfonts}

% used for TeXing text within eps files
%\usepackage{psfrag}
% need this for including graphics (\includegraphics)
%\usepackage{graphicx}
% for neatly defining theorems and propositions
%\usepackage{amsthm}
% making logically defined graphics
%%%\usepackage{xypic}

% there are many more packages, add them here as you need them

% define commands here
\begin{document}
Roughly, an orbifold is the quotient of a manifold by a finite group.  For example, take a sheet of paper and add a small crease perpendicular to one side at the halfway point.  Then, line up the two halves of the side.  This may be thought of as the plane $\mathbb R^2$ modulo the group $\mathbb Z^2$.  Now, let us give the definition.

Define a category $\mathcal X$:
The objects are pairs $(G,X)$, where $G$ is a finite group acting
effectively on a connected smooth manifold $X$.  A morphism $\Phi$ between
two objects $(G^\prime, X^\prime)$ and $(G,X)$ is a family of open
embeddings $\phi:X^\prime\rightarrow X$ which satisfy

\begin{itemize}
\item for each embedding $\phi\in\Phi$, there is an injective homomorphism
$\lambda_\phi:G^\prime\rightarrow G$ such that $\phi$ is $\lambda_\phi$
equivariant
\item For $g\in G$, we have
\begin{align*}
g\phi&:X^\prime\rightarrow X\\
g\phi&:x\mapsto g\phi(x)
\end{align*}
and if $(g\phi)(X) \cap \phi(X) \ne \emptyset$, then $g\in
\lambda_\phi(G^\prime)$.
\item $\Phi = \{g\phi, g\in G$\}, for any $\phi \in \Phi$
\end{itemize}



Now, we define orbifolds.  Given a paracompact Hausdorff space $X$ and a
nice open covering $\mathcal U$ which forms a basis for the topology on
$X$, an orbifold structure $\mathcal V$ on $X$ consists of
\begin{enumerate}
\item For $U\in \mathcal U$, $\mathcal V(U) = (G_U,\tilde U)\stackrel \tau
\rightarrow U$ is a ramified cover $\tilde U \rightarrow U$ which
identifies $\tilde U\slash G_U \cong U$
\item For $U\subset V \in \mathcal U$, there exists a morphism
$\phi_{VU}(G_U, \tilde U)\rightarrow (G_V, \tilde V)$ covering the
inclusion
\item If $U\subset V\subset W \in \mathcal U$, $\phi_{WU} =
\phi_{WV}\circ\phi_{VU}$
\end{enumerate}

References:

[1] Kawasaki T., The Signature theorem for V-manifolds. Topology 17 (1978), 75-83. MR0474432
(57:14072)
%%%%%
%%%%%
\end{document}
