\documentclass[12pt]{article}
\usepackage{pmmeta}
\pmcanonicalname{Transversality}
\pmcreated{2013-03-22 13:29:46}
\pmmodified{2013-03-22 13:29:46}
\pmowner{mathcam}{2727}
\pmmodifier{mathcam}{2727}
\pmtitle{transversality}
\pmrecord{6}{34071}
\pmprivacy{1}
\pmauthor{mathcam}{2727}
\pmtype{Definition}
\pmcomment{trigger rebuild}
\pmclassification{msc}{57R99}
\pmdefines{transversal}
\pmdefines{transverse}
\pmdefines{transversally}
\pmdefines{transversely}

% this is the default PlanetMath preamble.  as your knowledge
% of TeX increases, you will probably want to edit this, but
% it should be fine as is for beginners.

% almost certainly you want these
\usepackage{amssymb}
\usepackage{amsmath}
\usepackage{amsfonts}

% used for TeXing text within eps files
%\usepackage{psfrag}
% need this for including graphics (\includegraphics)
%\usepackage{graphicx}
% for neatly defining theorems and propositions
%\usepackage{amsthm}
% making logically defined graphics
%%%\usepackage{xypic}

% there are many more packages, add them here as you need them

% define commands here
\begin{document}
Transversality is a fundamental concept in differential topology. We say that two smooth submanifolds $A,B$ of a smooth manifold $M$ intersect \emph{transversely}, if at any point $x\in A\cap B$, we have
\[ T_x A + T_x B = T_x X, \]
where $T_x$ denotes the tangent space at $x$, and we naturally identify $T_x A$ and $T_x B$ with subspaces of $T_x X$. 

In this case, $A$ and $B$ intersect properly in the sense that $A\cap B$ is a submanifold of $M$, and
\[ \mathrm{codim}(A\cap B) = \mathrm{codim}(A) + \mathrm{codim}(B). \]

A useful generalization is obtained if we replace the inclusion $A\hookrightarrow M$ with a smooth map $f:A\to M$. In this case we say that $f$ is transverse to $B\subset M$, if for each point $a\in f^{-1}(B)$, we have
\[ df_a(T_a A) + T_{f(a)}B = T_{f(a)}M. \]
In this case it turns out, that $f^{-1}(B)$ is a submanifold of $A$, and
\[ \mathrm{codim}(f^{-1}(B)) = \mathrm{codim}(B).\]

Note that if $B$ is a single point $b$, then the condition of $f$ being transverse to $B$ is precisely that $b$ is a regular value for $f$. The result is that $f^{-1}(b)$ is a submanifold of $A$. A further generalization can be obtained by replacing the inclusion of $B$ by a smooth function as well. We leave the details to the reader. 

The importance of transversality is that it's a stable and generic condition. This means, in broad terms that if $f:A\to M$ is transverse to $B\subset M$, then small perturbations of $f$ are also transverse to $B$. Also, given any smooth map $A\to M$, it can be perturbed slightly to obtain a smooth map which is transverse to a given submanifold $B\subset M$.
%%%%%
%%%%%
\end{document}
