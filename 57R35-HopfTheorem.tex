\documentclass[12pt]{article}
\usepackage{pmmeta}
\pmcanonicalname{HopfTheorem}
\pmcreated{2013-03-22 14:52:34}
\pmmodified{2013-03-22 14:52:34}
\pmowner{jirka}{4157}
\pmmodifier{jirka}{4157}
\pmtitle{Hopf theorem}
\pmrecord{8}{36553}
\pmprivacy{1}
\pmauthor{jirka}{4157}
\pmtype{Theorem}
\pmcomment{trigger rebuild}
\pmclassification{msc}{57R35}

\endmetadata

% this is the default PlanetMath preamble.  as your knowledge
% of TeX increases, you will probably want to edit this, but
% it should be fine as is for beginners.

% almost certainly you want these
\usepackage{amssymb}
\usepackage{amsmath}
\usepackage{amsfonts}

% used for TeXing text within eps files
%\usepackage{psfrag}
% need this for including graphics (\includegraphics)
%\usepackage{graphicx}
% for neatly defining theorems and propositions
\usepackage{amsthm}
% making logically defined graphics
%%%\usepackage{xypic}

% there are many more packages, add them here as you need them

% define commands here
\theoremstyle{theorem}
\newtheorem*{thm}{Theorem}
\newtheorem*{lemma}{Lemma}
\newtheorem*{conj}{Conjecture}
\newtheorem*{cor}{Corollary}
\newtheorem*{example}{Example}
\theoremstyle{definition}
\newtheorem*{defn}{Definition}
\begin{document}
In the following we will assume that the term ``smooth'' implies just $C^1$ (once continuously differentiable).  By smooth homotopy we will \PMlinkescapetext{mean} that the homotopy mapping is itself continuously differentiable

\begin{thm}
Suppose that $M$ is a connected, \PMlinkname{oriented}{Orientation2} smooth manifold without boundary of dimension $m$ and suppose $f,g \colon M \to S^m$ are smooth mappings to the $m$-sphere.  Then $f$ and $g$ are smoothly homotopic if and only if $f$ and $g$ have the same Brouwer degree.
\end{thm}

When $M$ is not orientable, then we can always ``flip'' the orientation by following a closed loop on the manifold and one can then prove the following
result.

\begin{thm}
Suppose that $M$ is not orientable, connected smooth manifold without boundary
of dimension $m$, and suppose $f,g \colon M \to S^m$ are smooth mappings to
the $m$-sphere.
Then $f$ and $g$ are smoothly homotopic if and only if $f$ and $g$ have the
same degree mod 2.
\end{thm}

\begin{thebibliography}{9}
\bibitem{Milnor:topdiff}
John~W. Milnor.
{\em \PMlinkescapetext{Topology From The Differentiable Viewpoint}}.
The University Press of Virginia, Charlottesville, Virginia, 1969.
\end{thebibliography}
%%%%%
%%%%%
\end{document}
