\documentclass[12pt]{article}
\usepackage{pmmeta}
\pmcanonicalname{CohomologyOfCompactConnectedLieGroups}
\pmcreated{2013-03-22 17:49:47}
\pmmodified{2013-03-22 17:49:47}
\pmowner{asteroid}{17536}
\pmmodifier{asteroid}{17536}
\pmtitle{cohomology of compact connected Lie groups}
\pmrecord{10}{40295}
\pmprivacy{1}
\pmauthor{asteroid}{17536}
\pmtype{Feature}
\pmcomment{trigger rebuild}
\pmclassification{msc}{57T10}
\pmclassification{msc}{58A12}
\pmclassification{msc}{55N99}
\pmclassification{msc}{22E15}
\pmrelated{InvariantDifferentialForm}

\endmetadata

% this is the default PlanetMath preamble.  as your knowledge
% of TeX increases, you will probably want to edit this, but
% it should be fine as is for beginners.

% almost certainly you want these
\usepackage{amssymb}
\usepackage{amsmath}
\usepackage{amsfonts}

% used for TeXing text within eps files
%\usepackage{psfrag}
% need this for including graphics (\includegraphics)
%\usepackage{graphicx}
% for neatly defining theorems and propositions
%\usepackage{amsthm}
% making logically defined graphics
%%%\usepackage{xypic}

% there are many more packages, add them here as you need them

% define commands here

\begin{document}
\PMlinkescapephrase{properties}
\PMlinkescapephrase{structure}
\PMlinkescapephrase{type}
\PMlinkescapephrase{singular}
\PMlinkescapephrase{theory}
\PMlinkescapephrase{proposition}
\PMlinkescapephrase{compact}
\PMlinkescapephrase{connected}
\PMlinkescapephrase{cohomology}
\PMlinkescapephrase{complex}
\PMlinkescapephrase{right}
\PMlinkescapephrase{invariant}
\PMlinkescapephrase{tangent space}
\PMlinkescapephrase{adjoint}
\PMlinkescapephrase{alternating}

This entry aims to describe some properties of the \PMlinkname{cohomology}{DeRhamCohomology} of \PMlinkname{compact}{Compact} \PMlinkname{connected}{ConnectedSpace} Lie groups. It turns out that this type of Lie groups admit a somewhat simplified cohomology theory, compared with singular cohomology or de Rham cohomology. This simplified theory then allows one to observe some of the \PMlinkescapetext{restrictions} imposed on the structure of the cohomology groups of compact connected Lie groups.

The construction and results presented in the entry on invariant differential forms are assumed and will be the \PMlinkescapetext{base} of what we are about to describe.

\section{Cohomology of invariant forms}

Let $G$ be a compact connected Lie group. Given a differential manifold $M$ and smooth action of $G$ in $M$, let $\Omega_G^k(M)$ denote the space of the \PMlinkname{$k$-invariant differential forms}{InvariantDifferentialForm} in $M$. It is known (see \PMlinkname{this entry}{InvariantDifferentialForm}) that $\Omega_G^*(M)$ forms a chain complex and the cohomology groups of this complex are isomorphic to the cohomology groups of $M$, i.e.
\begin{displaymath}
H^k(\Omega_G(M)) \cong H^k(M;\mathbb{R})
\end{displaymath}

We now regard the manifold $M$ as a compact connected Lie group $G$. There are several smooth actions on $G$ that are worth to be considered, such as:
\begin{itemize}
\item The action of $G$ on itself by left multiplication.
\item The action of $G$ on itself by right multiplication.
\item The action of $G$ on itself by conjugation.
\item The action of $G\times G$ on $G$ given by $(g,h)\cdot k :=gkh^{-1}$
\end{itemize}

Thus, by the previous remark, the cohomology of a compact connected Lie group $G$ restricts to the cohomology of left invariant forms, right invariant forms, adjoint invariant forms or \PMlinkname{bi-invariant forms}{InvariantDifferentialForm}. Moreover, the cohomology of $G$ is the cohomology of differential forms invariant under any smooth action of any compact connect Lie group on $G$ .

\subsection{Left invariant forms}

Since left invariant forms are uniquely determined by their values in $T_eG$, the \PMlinkname{tangent space}{TangentSpace} at the identity, they allow a simpler characterization of the cohomology of compact connected Lie groups.

$\,$

{\bf Proposition -} Let $G$ be a compact connected Lie group and $T_eG^*$ the chain complex of alternating forms on $T_eG$ with coboundary operator $d$ given by
\begin{displaymath}
d\omega (X_0, \dots , X_k) = \sum_{i<j} (-1)^{i+j} \omega([X_i, X_j], X_0, \dots, \hat{X_i}, \dots, \hat{X_j}, \dots, X_k)
\end{displaymath}
Then, the cohomology groups of $H^k(G, \mathbb{R})$ are isomorphic to the cohomology groups $H^k(T_eG^*)$ of the complex $T_eG^*$.

$\,$

\subsection{Bi-invariant forms}
Two sided invariance is just the same as left invariance \PMlinkescapetext{plus} adjoint invariance. Hence, every bi-invariant form can be seen as an alternating form on $T_eG$ wich is also adjoint invariant. Thus, just like the previous proposition, the cohomology groups of $G$ are just the cohomology groups of the complex $T_eG^*_{ad}$ of adjoint invariant alternating forms on $T_eG$, with coboundary operator
\begin{displaymath}
d\omega (X_0, \dots , X_k) = \sum_{i<j} (-1)^{i+j} \omega([X_i, X_j], X_0, \dots, \hat{X_i}, \dots, \hat{X_j}, \dots, X_k)
\end{displaymath}

This can be further improved. The following important theorem will be the key to more specific results.

$\,$

{\bf Theorem 1 -} Let $G$ be a compact connected Lie group. Let $\omega$ be a multilinear $k$-form on $T_e$. Then $\omega$ is adjoint invariant if and only if the following equality holds:
\begin{displaymath}
\sum_{i=1}^k \omega(X_1, \dots, X_{i-1}, [Y,X_i],X_{i+1},\dots,X_k)=0 \;\;\;\;\; \text{for all}\; Y, X_1, \dots, X_k \in T_eG
\end{displaymath}

$\,$

{\bf Proposition -} Every adjoint invariant alternating form on $T_eG$ is \PMlinkname{closed}{ClosedDifferentialForm}.

{\bf \emph{Proof:}} Let $\alpha_{i,i}=0$, $\alpha_{i,j}=(-1)^j$ if $i <j$ and $\alpha_{i,j} = (-1)^{j+1}$ if $i > j$. Then
\begin{eqnarray*}
d\omega (X_0, \dots , X_k) & = & \frac{1}{2} \sum_{i \neq j} (-1)^i \alpha_{i,j} \omega([X_i, X_j], X_0, \dots, \hat{X_i}, \dots, \hat{X_j}, \dots, X_k)\\
& = & \frac{1}{2} \sum_i (-1)^i \sum_j \alpha_{i,j} \omega([X_i, X_j], X_0, \dots, \hat{X_i}, \dots, \hat{X_j}, \dots, X_k)\\
& = & 0 \,
\end{eqnarray*}
where the \PMlinkescapetext{inner} sum is zero by Theorem 1. $\square$

$\,$

{\bf Corollary 1 -} Let $G$ be a compact connected Lie group. The cohomology groups $H^k(G;\mathbb{R})$ are isomorphic to the vector space of adjoint invariant alternating $k$-forms on $T_eG$.

$\,$

\section{Relations between cohomology groups}

Let $G$ be a compact connected Lie group. In the proofs of the following results we are always using the fact stated in Corollary 1, that $H^k(G;\mathbb{R})$ is exactly the space of adjoint invariant alternating $k$-forms on $T_eG$. Let $[T_eG, T_eG]$ denote the \PMlinkname{subspace}{VectorSubspace} of $T_eG$ generated by elements of the form $[X,Y]$, with $X, Y \in T_eG$.

$\;$

{\bf Proposition -} $\qquad [T_eG, T_eG] = T_eG \;\, \Longleftrightarrow \;\, H^1(G;\mathbb{R})=0$

{\bf \emph{Proof:}} It follows easily form the fact that adjoint invariant alternating $1$-forms on $T_eG$ are precisely those who satisfy $\omega([X,Y]) = 0$ for all $X, Y \in T_eG$ (Theorem 1). $\square$

$\,$

{\bf Corollary -} $\qquad H^1(G;\mathbb{R}) = 0 \;\, \Longrightarrow \;\, H^2(G;\mathbb{R})=0$

{\bf \emph{Proof:}} Let $\omega$ be an adjoint invariant $2$-form on $T_eG$. We have that
\begin{eqnarray*}
0 = d\omega(X, Y, Z) & = & -\omega([X,Y],Z) + \omega([X,Z],Y) - \omega([Y,Z],X)\\
& = & -\omega([X,Y],Z) - \Big( \omega([Z, X],Y) + \omega(X,[Z,Y]) \Big)\\
& = & -\omega([X,Y],Z)
\end{eqnarray*}
where the last step comes from Theorem 1. Now, as $[T_eG, T_eG] = T_eG$ by the previous Proposition, $\omega = 0$. $\square$

$\,$

Let $Sym^2$ be the space of adjoint invariant symmetric bilinear forms on $T_eG$ and $T_eG_ad^k$ the space of adjoint invariant alternating $k$-forms on $T_eG$.

$\,$

{\bf Theorem 2 - } Suppose $H^1(G;\mathbb{R})=0$. The function $\Phi : Sym^2 \longrightarrow T_eG^3$ defined by
\begin{displaymath}
\Phi(\eta)\,(X,Y,Z) := \eta([X,Y],Z)\,, \qquad\qquad X,Y,Z \in T_eG
\end{displaymath}
is well defined and bijective.

$\,$

One can always assure the existence of nonzero adjoint invariant symmetric bilinear forms on $T_eG$. This can be achieved by taking an inner product $\langle \cdot , \cdot \rangle$ on $T_eG$ and defining, for $X,Y \in T_eG$,
\begin{displaymath}
\eta(X,Y):= \frac{1}{\mu(G)}\int_G \langle (C_g)_* X , (C_g)_* Y \rangle \;d\mu(g)
\end{displaymath}
where $\mu$ is the Haar measure of $G$ and $C_g$ is the function of conjugation by $g \in G$. Hence we can conclude that

$\,$

{\bf Corollary -} Suppose $G$ is nontrivial. Then $\;\; H^1(G;\mathbb{R}) = 0 \;\, \Longrightarrow \;\, H^3(G;\mathbb{R})\neq0$


%%%%%
%%%%%
\end{document}
