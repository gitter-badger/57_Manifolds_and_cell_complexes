\documentclass[12pt]{article}
\usepackage{pmmeta}
\pmcanonicalname{BrouwerDegree}
\pmcreated{2013-03-22 14:52:37}
\pmmodified{2013-03-22 14:52:37}
\pmowner{jirka}{4157}
\pmmodifier{jirka}{4157}
\pmtitle{Brouwer degree}
\pmrecord{7}{36554}
\pmprivacy{1}
\pmauthor{jirka}{4157}
\pmtype{Definition}
\pmcomment{trigger rebuild}
\pmclassification{msc}{57R35}
\pmsynonym{degree}{BrouwerDegree}
\pmrelated{DegreeMod2OfAMapping}

% this is the default PlanetMath preamble.  as your knowledge
% of TeX increases, you will probably want to edit this, but
% it should be fine as is for beginners.

% almost certainly you want these
\usepackage{amssymb}
\usepackage{amsmath}
\usepackage{amsfonts}

% used for TeXing text within eps files
%\usepackage{psfrag}
% need this for including graphics (\includegraphics)
%\usepackage{graphicx}
% for neatly defining theorems and propositions
\usepackage{amsthm}
% making logically defined graphics
%%%\usepackage{xypic}

% there are many more packages, add them here as you need them

% define commands here
\theoremstyle{theorem}
\newtheorem*{thm}{Theorem}
\newtheorem*{lemma}{Lemma}
\newtheorem*{conj}{Conjecture}
\newtheorem*{cor}{Corollary}
\theoremstyle{definition}
\newtheorem*{defn}{Definition}
\begin{document}
Suppose that $M$ and $N$ are two oriented differentiable manifolds
of dimension $n$ (without boundary) with $M$ compact and $N$ connected and suppose that
$f \colon M \to N$ is a differentiable mapping.  Let $Df(x)$ denote the
differential mapping at the point $x \in M$,
that is the linear mapping $Df(x) \colon T_x(M) \to T_{f(x)}(N)$.  Let $\operatorname{sign} Df(x)$ denote the sign
of the determinant of $Df(x)$.  That is the sign is positive if $f$ preserves
orientation and negative if $f$ reverses orientation.

\begin{defn}
Let $y \in N$ be a regular value, then we define the {\em Brower degree} (or just
degree) of $f$ by
\begin{equation*}
\operatorname{deg} f := \sum_{x \in f^{-1}(y)} \operatorname{sign} Df(x) .
\end{equation*}
\end{defn}

It can be shown that the degree does not depend on the regular value $y$ that we pick so that $\operatorname{deg} f$ is well defined.

Note that this degree coincides with the \PMlinkname{degree}{Degree5} as defined for maps of spheres.

\begin{thebibliography}{9}
\bibitem{Milnor:topdiff}
John~W. Milnor.
{\em \PMlinkescapetext{Topology From The Differentiable Viewpoint}}.
The University Press of Virginia, Charlottesville, Virginia, 1969.
\end{thebibliography}
%%%%%
%%%%%
\end{document}
