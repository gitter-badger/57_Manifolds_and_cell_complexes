\documentclass[12pt]{article}
\usepackage{pmmeta}
\pmcanonicalname{ExamplesOfMappingClassGroup}
\pmcreated{2013-03-22 15:41:19}
\pmmodified{2013-03-22 15:41:19}
\pmowner{juanman}{12619}
\pmmodifier{juanman}{12619}
\pmtitle{examples of mapping class group}
\pmrecord{8}{37631}
\pmprivacy{1}
\pmauthor{juanman}{12619}
\pmtype{Example}
\pmcomment{trigger rebuild}
\pmclassification{msc}{57R50}
\pmsynonym{first homeotopy group}{ExamplesOfMappingClassGroup}
%\pmkeywords{Dehn's twist}
%\pmkeywords{surface}
\pmrelated{isotopy}
\pmrelated{group}
\pmrelated{Group}
\pmrelated{Isotopy}

\endmetadata

% this is the default PlanetMath preamble.  as your knowledge
% of TeX increases, you will probably want to edit this, but
% it should be fine as is for beginners.

% almost certainly you want these
\usepackage{amssymb}
\usepackage{amsmath}
\usepackage{amsfonts}

% used for TeXing text within eps files
%\usepackage{psfrag}
% need this for including graphics (\includegraphics)
%\usepackage{graphicx}
% for neatly defining theorems and propositions
%\usepackage{amsthm}
% making logically defined graphics
%%%\usepackage{xypic}

% there are many more packages, add them here as you need them

% define commands here
\begin{document}
An example of this concept is to take the 2-sphere $S^2$, then one can calculate that
$${\cal{M}}(S^2)=1,$$
but 
$${\cal{M}}^*(S^2)={\mathbb{Z}}_2.$$

For the genus one orientable surface, i.e. the torus $T=S^1\times S^1$, it is known that its (extended) mapping class group 
$${\cal{M}}^*(T)=GL_2({\mathbb{Z}}),$$ 
but usually by the (non-extended) mapping class group, that is, the group of isotopy classes of homeomorphisms that preserve orientations (the Dehn's twists) is just
$${\cal{M}}(T)=SL_2({\mathbb{Z}}).$$

In these two examples we see that $\cal{M}^*$ is an extension of $\cal{M}$ by ${\mathbb{Z}}_2$, trivial for the 2-sphere and non trivial for the torus.

For the projective plane ${\mathbb{R}}P^2$ we have 
$${\cal{M}}({\mathbb{R}}P^2)={\cal{M}}^*({\mathbb{R}}P^2)=1$$

And what about the Klein bottle?
$${\cal{M}}(K)={\mathbb{Z}}_2$$
$${\cal{M}}^*(K)={\mathbb{Z}}_2\oplus{\mathbb{Z}}_2$$
%%%%%
%%%%%
\end{document}
