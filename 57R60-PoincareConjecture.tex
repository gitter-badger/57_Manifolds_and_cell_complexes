\documentclass[12pt]{article}
\usepackage{pmmeta}
\pmcanonicalname{PoincareConjecture}
\pmcreated{2013-03-22 13:56:16}
\pmmodified{2013-03-22 13:56:16}
\pmowner{yark}{2760}
\pmmodifier{yark}{2760}
\pmtitle{Poincare conjecture}
\pmrecord{20}{34698}
\pmprivacy{1}
\pmauthor{yark}{2760}
\pmtype{Conjecture}
\pmcomment{trigger rebuild}
\pmclassification{msc}{57R60}


\begin{document}
\PMlinkescapeword{elementary}
\PMlinkescapeword{eventually}
\PMlinkescapeword{information}
\PMlinkescapeword{period}
\PMlinkescapeword{proof}
\PMlinkescapeword{series}

Until its proof in 2003,
the Poincar\'e Conjecture was the central problem of low-dimensional topology.
It is one of the
\PMlinkexternal{Clay Mathematics Institute}{http://www.claymath.org}
\PMlinkexternal{Millennium Prize Problems}{http://www.claymath.org/millennium/},
and so far the only one to be solved.

{\bf Theorem}
\emph{Every 3-manifold without boundary
that is homotopy equivalent to the $3$-sphere
is in fact homeomorphic to it.}
Or, in a more elementary form:
\emph{every simply-connected compact $3$-manifold without boundary
is homeomorphic to $S^3$}.

The first statement is also true
when $3$ is replaced by any other positive integer,
but the 3-dimensional case turned out to be much harder than the other cases.

The Poincar\'e Conjecture was eventually proved by Grigoriy Perelman,
who gave his proof in a series of preprints posted on arXiv.org in 2002 and 2003.
For this work Perelman was offered a Fields Medal in 2006, though he declined it.
On 18 March 2010, he was awarded the \$1,000,000 Millennium Prize
for resolution of the Poincar\'e Conjecture, but has also declined this prize.
(The long delay in the awarding of the Millennium Prize
was largely due to the way that Perelman chose to publish his results.
Full details of his proof did not appear
in a peer reviewed mathematical publication until 2006,
and the Millennium Prize rules require a waiting period of two years
after such publication before the prize can be awarded.)

See also \PMlinkexternal{Clay Mathematics Institute press release on the Millennium Prize for the Poincar\'e Conjecture}{http://www.claymath.org/poincare/millenniumPrizeFull.pdf} (PDF file),
and the \PMlinkexternal{Clay Mathematics Institute's page on Perelman's work}{http://www.claymath.org/millennium/Poincare_Conjecture/perelman+expositions.php}.
%%%%%
%%%%%
\end{document}
