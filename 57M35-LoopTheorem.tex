\documentclass[12pt]{article}
\usepackage{pmmeta}
\pmcanonicalname{LoopTheorem}
\pmcreated{2013-03-22 15:49:13}
\pmmodified{2013-03-22 15:49:13}
\pmowner{juanman}{12619}
\pmmodifier{juanman}{12619}
\pmtitle{loop theorem}
\pmrecord{11}{37786}
\pmprivacy{1}
\pmauthor{juanman}{12619}
\pmtype{Theorem}
\pmcomment{trigger rebuild}
\pmclassification{msc}{57M35}
%\pmkeywords{embeddings}
%\pmkeywords{fundamental group}
\pmrelated{3Manifolds}

\endmetadata

% this is the default PlanetMath preamble.  as your knowledge
% of TeX increases, you will probably want to edit this, but
% it should be fine as is for beginners.

% almost certainly you want these
\usepackage{amssymb}
\usepackage{amsmath}
\usepackage{amsfonts}

% used for TeXing text within eps files
%\usepackage{psfrag}
% need this for including graphics (\includegraphics)
%\usepackage{graphicx}
% for neatly defining theorems and propositions
%\usepackage{amsthm}
% making logically defined graphics
%%%\usepackage{xypic}

% there are many more packages, add them here as you need them

% define commands here
\begin{document}
In the topology of 3-manifolds, {\bf the loop theorem} is generalization of an ansatz discovered by Max Dehn (namely, Dehn's lemma), 
who saw that {\em if a continuous map from a 2-disk to a 3-manifold whose restriction to the boundary's disk has no singularities, 
then there exists another embedding whose restriction to the boundary's disk is equal to the boundary's restriction original map}.

The following statement called the loop theorem is a version from J. Stallings, but written in W. Jaco's book.

{\em Let $M$ be a three-manifold and let $S$
 be a connected surface in $\partial M $. Let $N\subset \pi_1(M)$ be a normal subgroup.
Let  $f \colon D^2\to M $
be a {\bf continuous map} such that $f(\partial D^2)\subset S$
and $[f|\partial D^2]\notin N$.\\
Then there exists an {\bf embedding} 
$g\colon D^2\to M $ such that 
$g(\partial D^2)\subset S$
and 
$[g|\partial D^2]\notin N$},

The proof is a clever construction due to C. Papakyriakopoulos about a sequence (a tower) of covering spaces.
Maybe the best detailed presentation is due to A. Hatcher.
But in general, accordingly to  Jaco's opinion, {\it ''... for anyone unfamiliar with the techniques of 3-manifold-topology and are here to gain a working knowledge  for the study of problems in this 
\PMlinkescapeword{area}..., there is no better 
\PMlinkescapeword{place} to start.''}



{\bf References}

W. Jaco, {\it Lectures on 3-manifolds topology}, A.M.S. regional conference series in Math 43.

J. Hempel, {\it 3-manifolds}, Princeton University Press 1976.

A. Hatcher, {\it Notes on 3-manifolds}, available on-line.
%%%%%
%%%%%
\end{document}
