\documentclass[12pt]{article}
\usepackage{pmmeta}
\pmcanonicalname{3manifold}
\pmcreated{2013-03-22 15:40:55}
\pmmodified{2013-03-22 15:40:55}
\pmowner{juanman}{12619}
\pmmodifier{juanman}{12619}
\pmtitle{3-manifold}
\pmrecord{17}{37623}
\pmprivacy{1}
\pmauthor{juanman}{12619}
\pmtype{Definition}
\pmcomment{trigger rebuild}
\pmclassification{msc}{57N10}
%\pmkeywords{Seifert fiber spaces}
%\pmkeywords{bundle}
\pmrelated{manifold}
\pmrelated{DehnsLemma}
\pmrelated{SphereTheorem}
\pmrelated{LoopTheorem}
\pmrelated{SeifertFiberSpace}
\pmrelated{Manifold}

% this is the default PlanetMath preamble.  as your knowledge
% of TeX increases, you will probably want to edit this, but
% it should be fine as is for beginners.

% almost certainly you want these
\usepackage{amssymb}
\usepackage{amsmath}
\usepackage{amsfonts}

% used for TeXing text within eps files
%\usepackage{psfrag}
% need this for including graphics (\includegraphics)
%\usepackage{graphicx}
% for neatly defining theorems and propositions
%\usepackage{amsthm}
% making logically defined graphics
%%%\usepackage{xypic}

% there are many more packages, add them here as you need them

% define commands here
%\usepackage{bbm}
%\newcommand{\Z}{\mathbbmss{Z}}
%\newcommand{\C}{\mathbbmss{C}}
%\newcommand{\R}{\mathbbmss{R}}
%\newcommand{\Q}{\mathbbmss{Q}}
%\newcommand{\mathbb}[1]{\mathbbmss{#1}}
\begin{document}
In this brief note we define and give instances of the notion of a 3-manifold.

A \emph{3-manifold} is a Hausdorff topological space which is locally homeomorphic to the Euclidean space ${\mathbb{R}}^3$.

One can see from simple constructions the great variety of objects that indicate how they are worth to study.

First examples without boundary:
\begin{enumerate}
\item For example, with the Cartesian product we can get:
\begin{itemize}
\item $S^2\times S^1$
\item ${\mathbb{R}}P^2\times S^1$
\item $T\times S^1$
\item $K\times S^1$
\end{itemize}
where $S^1$ and $S^2$ are the 1- and 2-dimensional spheres respectively, $T$ is a torus, $K$ a Klein bottle, and $\mathbb{R}P^2$ is the 2-dimensional real projective space.

\item Also by the generalization of the Cartesian product: \emph{fiber bundles}, one can build bundles $E$ of the type
$$F\subset E\to S^1$$
where $F$ is any closed surface.

\item Or interchanging the roles, bundles as:
$$S^1\subset E\to F$$

\item knots and links complements
\end{enumerate}

For the second type it is known that for each \emph{isotopy class} $[\phi]$ of maps $F\to F$ correspond to an unique bundle $E_{\phi}$. Any homeomorphism $f:F\to F$ representing the isotopy class $[\phi]$ is called a \emph{monodromy} for $E_{\phi}$.

From the previous paragraph we infer that the \emph{mapping class group} play a important role in the understanding at least for this subclass of objets.

For the third class above one can use an \emph{orbifold} instead of a simple surface to get a class of 3-manifolds called \emph{Seifert fiber spaces} which are a large class of spaces needed to understand the modern classifications for 3-manifolds.


{\bf References}
\begin{itemize}
\item  J.C. G\'omez-Larra\~naga. {\it 3-manifolds which are unions of three solid tori},
Manuscripta Math. 59 (1987), 325-330.
\item  J.C. G\'omez-Larra\~naga, F.J. Gonz\'alez-Acu\~na, J. Hoste.
{\it Minimal Atlases on 3-manifolds},
Math. Proc. Camb. Phil. Soc. 109 (1991), 105-115.
\item  J. Hempel. {\it 3-manifolds}, Princeton University Press 1976.
\item  P. Orlik. {\it Seifert Manifolds}, Lecture Notes in Math. 291,
1972 Springer-Verlag.
\item  P. Scott. {\it The geometry of 3-manifolds}, Bull. London Math. Soc. 15 (1983), 401-487.
\end{itemize}
%%%%%
%%%%%
\end{document}
