\documentclass[12pt]{article}
\usepackage{pmmeta}
\pmcanonicalname{FunctionDifferentiableAtOnlyOnePoint}
\pmcreated{2013-03-22 15:48:16}
\pmmodified{2013-03-22 15:48:16}
\pmowner{matte}{1858}
\pmmodifier{matte}{1858}
\pmtitle{function differentiable at only one point}
\pmrecord{6}{37767}
\pmprivacy{1}
\pmauthor{matte}{1858}
\pmtype{Example}
\pmcomment{trigger rebuild}
\pmclassification{msc}{57R35}
\pmclassification{msc}{26A24}
\pmrelated{FunctionContinuousAtOnlyOnePoint}

\endmetadata

% this is the default PlanetMath preamble.  as your knowledge
% of TeX increases, you will probably want to edit this, but
% it should be fine as is for beginners.

% almost certainly you want these
\usepackage{amssymb}
\usepackage{amsmath}
\usepackage{amsfonts}
\usepackage{amsthm}

\usepackage{mathrsfs}

% used for TeXing text within eps files
%\usepackage{psfrag}
% need this for including graphics (\includegraphics)
%\usepackage{graphicx}
% for neatly defining theorems and propositions
%
% making logically defined graphics
%%%\usepackage{xypic}

% there are many more packages, add them here as you need them

% define commands here

\newcommand{\sR}[0]{\mathbb{R}}
\newcommand{\sC}[0]{\mathbb{C}}
\newcommand{\sN}[0]{\mathbb{N}}
\newcommand{\sZ}[0]{\mathbb{Z}}

 \usepackage{bbm}
 \newcommand{\Z}{\mathbbmss{Z}}
 \newcommand{\C}{\mathbbmss{C}}
 \newcommand{\F}{\mathbbmss{F}}
 \newcommand{\R}{\mathbbmss{R}}
 \newcommand{\Q}{\mathbbmss{Q}}



\newcommand*{\norm}[1]{\lVert #1 \rVert}
\newcommand*{\abs}[1]{| #1 |}



\newtheorem{thm}{Theorem}
\newtheorem{defn}{Definition}
\newtheorem{prop}{Proposition}
\newtheorem{lemma}{Lemma}
\newtheorem{cor}{Corollary}
\begin{document}
Let $f\colon \R\to \R$ be the function 
$$
f(x) = \begin{cases} x, & \mbox{when $x$ is rational}, \\
-x, & \mbox{when $x$ is irrational}.
\end{cases}
$$
See \PMlinkname{this entry}{FunctionContinuousAtOnlyOnePoint}.
Let $g\colon \R\to \R$ be the function
$$ 
  g(x) = f(x) x.
$$
Then $g$ differentiable at $0$,
but everywhere else non-differentiable.

Indeed, since
\begin{eqnarray*}
  g'(0) &=& \lim_{h\to 0} \frac{f(h)h-f(0)0}{h} \\
        &=& \lim_{h\to 0} f(h) \\
        &=& 0
\end{eqnarray*}
$g$ is differentiable at $0$. 
If $g$ would be continuous at $x\neq 0$, then $f(x)=g(x)/x$ 
would be continuous at $x$. 
\PMlinkname{This result}{DifferentiableFunctionsAreContinuous}
implies that $g$ is non-differentiable away from the origin.
%%%%%
%%%%%
\end{document}
